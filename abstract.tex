Pattern mining in spatio-temporal datasets is a really relevant subject in the academia and the industry nowadays, due
to its wide applicability in helping to solve real-world problems. Many of them can be found in the context of Smart
Cities, like Traffic Management, Surveillance and Security and City Planning, to name a few. Among the various
spatio-temporal patterns that one can extract from a spatio-temporal dataset, the flock pattern is one that has gained a
lot of attention, because of its intrinsic relation with the aforementioned problems. A lot of work has been done in the
academia, in order to provide algorithms able to identify the flock pattern. However, none of them could perform that
task efficiently nor be able to scale well when a large dataset was the analysis target. Additionally, we found that
there was no System Architecture proposal that could be simple and extensible enough to be used in the various
spatio-temporal pattern detection problems present today. Given that context, this dissertation proposes a Generic
System Archicture that can be used to solve a great number of spatio-temporal pattern mining problems. We then validate
such architecture by implementing on top of it an efficient flock detection algorithm, aiming at achieving considerable
gains in execution time without compromising accuracy, thus targeting real-time deployment and online processing in
Smart Cities. Last, but not least, we remodel our algorithm in order to take advantage of multi-core architectures
present in modern computers. Our results indicate that our proposal outperforms the current state-of-the-art techniques,
by achieving 99\% CPU time improvement. Moreover, with our multi-thread model, we were able to reduce the processing
time of our proposed algorithm by 96\% in some cases. We prove the efficiency of our solution by performing evaluation
with both real and synthetic large datasets.

\begin{keywords}
Smart Cities, Pattern Mining, Spatio-temporal data, Flock Pattern
\end{keywords}
