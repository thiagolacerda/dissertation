Detecção de padrões em dados espaço-temporais tem se mostrado um tema de muita relevância nos dias atuais, tanto na
academia quanto na indústria, devido a sua vasta aplicabilidade em auxiliar a solucionar problemas enfrentados na
sociedade. Muitos desses problemas podem ser classificados no conexto de Cidades Inteligentes (\textit{Smart Cities}),
como Gerenciamento de Tráfego, Segurança e Planejamento de Cidades. Dentre os vários padrões espaço-temporais que podem
ser extraídos de uma base de dados, o padrão de \textit{flock} é um que vem atraindo muita atenção, devido a sua relação
intrínseca com os problemas mencionados anteriormente. Muitas pesquisas vêm sendo feitas academia, visando desenvolver
algoritmos capazes de identificar o padrão de \textit{flock}. Porém, nenhum deles foi capaz de executar tal tarefa
eficientemente, nem conseguiu escalar de maneira aceitável quando uma base de dados de grande tamanho foi analisada.
Além disso, não foi encontrado nos trabalhos relacionados uma arquitetura de \textit{software} que conseguisse ser
simples e extensível o suficiente para ser usada nos vários problemas de detecção de padrões em dados espaço-temporais.
Com isso em mente, essa dissertação propõe uma Arquitetura de \textit{Software} que pode ser usada como base para
solucionar um grande número de problemas relacionados a mineração de padrões em dados espaço-temporais. Tal arquitetura
foi então usada como base na implementação de um algoritmo de detecção de \textit{flock}, focando em alcançar grandes
ganhos em tempo de processamento, sem comprometer a precisão, visando então cenários de aplicações de tempo real em
Cidades Inteligentes. No fim, nós propomos uma remodelagem no nosso algoritmo para poder utilizar ao máximo o poder de
processamento oferecido pelas arquiteturas Multi-core dos processadores modernos. Nossos resultados mostraram que nossa
solução conseguiu superar propostas do estado da arte, alcançando 99\% de redução no tempo de processamento total. Além
disso, nossa remodelagem \textit{multi-thread} conseguiu melhorar os resultados da nossa solução em até 96\% em alguns
casos. A eficiência e performance da nossa proposta foi comprovada com avaliações feitas com bases de dados geradas
sinteticamente e coletadas em experimentos reais.

\begin{keywords}
Cidades Inteligentes, Mineração de Padrões, Dados Espaço-temporais, Padrão de Flock
\end{keywords}
