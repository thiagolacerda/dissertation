\chapter{Conclusion}
\label{chp:conclusion}
Pattern mining in spatio-temporal datasets is a really relevant subject in the academy and the industry nowadays, due
to its wide applicability in helping to solve real world problems. Many of them can be found in the context of Smart
Cities, like Traffic Management, Surveillance and Security and City Planning, to name a few. Among the various
spatio-temporal patterns that one can extract from a spatio-temporal dataset, the Flock pattern is one that has gained a
lot of attention. The reason for such attention it because of its collective behavior properties, which are very
applicable and are intrinsically related to the type of problems that we have just mentioned.

Throughout this dissertation, we could show that a lot of work has been done in the academy, in order to provide
algorithms able to identify the Flock pattern. However, none of them could perform that task efficiently nor be able to
scale well when a large dataset was the analysis target. Additionally, we found that there was no system architecture
proposal that could be simple and extensible enough to be used in the various data mining problems present today. Hence,
since we are witnessing the rising of Smart Cities, helping to solve those aforementioned issues would be of tremendous
importance.

Given the importance of efficiently discovering Flock patterns, in order to enable decision makers to act fast, we
proposed a simple, extensible and reusable System Architecture that can fit in almost any data mining problem. We then
used that architecture to implement a novel flock pattern detection algorithm that was able to outperform the
state-of-the-art solutions in more than 99\%. Such impressive results were possible due to the filtering heuristic based
on bitmaps that we proposed, which was able to reduce the generation of cluster disks in more than 96\%, directly
reducing the amount of data that needed to be processed by the algorithm. Despite the great results achieved with our
solution, we realized that there was still room for improvements, given the large availability of multi-core processors
in today's computers. With that in mind, we remodeled our proposed solution, proving how flexible our proposed
architecture was, to perform some critical and time consuming tasks in parallel, taking full advantage of the multi-core
paradigm. Our performance benchmarks showed that we could outperform our own single threaded solution by 96\% in some
cases. So, to put that in numbers, we were able to reduce a running time of 15 thousand seconds (state-of-the-art
techniques) to only 13 seconds (BitDF MT). It is also important to mention that our experiments were performed using
various datasets, both synthetic generated and collected from real world experiments and all of them having a
considerable number of entries.

Even though we ere able to present great contributions in this field, that are still some gaps that can be filled and
points that need more work and can lead to important results as well. Here is a list with some of them:

\begin{enumerate}
    \item The disk superset and duplicate check is very costly and would need further investigation in order to make it
        more efficient. Additionally because it modifies the same disk set it is really hard to parallelize it without
        causing performance degradation.
    \item When investigating a grid cell, its extended grid might have cells that were previously processed by other
        extended grids. Thus, we could avoid paring the same points and generating the same disks again if we could keep
        some sort of disk cache and reuse those disks.
    \item There is no study to see how different flocks relate to each other, like some MO that starts in one flock and
        later moves to a different one.
    \item Make each module of the proposed architecture run in its own thread/process and then making it independent of
        the time spent in other modules.
    \item See how BitDF performs in indoor flock detection.
\end{enumerate}
% * <steniofernandes@gmail.com> 2016-05-26T15:15:43.547Z:
%
% add subsection or paragraph: deployment challenges in real-world scenarios... other prospective scenarios (e.g., emergency routes in case of evacuation , large crowds (indoor, outdoor), combination with voronoi diagram (google  'voronoi diagram escape evacuation' to get some ideas
%
% ^.
% * <steniofernandes@gmail.com> 2016-05-26T15:14:41.543Z:
%
% add future work subsection
%
% ^.