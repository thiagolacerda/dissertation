\chapter{Conclusion}
\label{chp:conclusion}
Pattern mining in spatio-temporal datasets is a really relevant subject in the academia and the industry nowadays, due
to its wide applicability in helping to solve real world problems. Many of them can be found in the context of Smart
Cities, like Traffic Management, Surveillance and Security and City Planning, to name a few. Among the various
spatio-temporal patterns that one can extract from a spatio-temporal dataset, the Flock pattern is one that has gained a
lot of attention. The reason for such attention it because of its collective behavior properties, which are very
applicable and are intrinsically related to the type of problems that we have just mentioned.

Throughout this dissertation, we could show that a lot of work has been done in the academia, in order to provide
algorithms able to identify the Flock pattern. However, none of them could perform that task efficiently nor be able to
scale well when a large dataset was the analysis target. Additionally, we found that there was no system architecture
proposal that could be simple and extensible enough to be used in the various spatio-temporal pattern detection problems
present today. Hence, since we are witnessing the rising of Smart Cities, helping to solve those aforementioned issues
would be of tremendous importance.

Given the importance of efficiently discovering Flock patterns, in order to enable decision makers to act fast, we
proposed a simple, extensible and reusable System Architecture that can fit in almost any spatio-temporal pattern
detection problem. We then used that architecture to implement a novel flock pattern detection algorithm that was able
to outperform the state-of-the-art solutions in more than 99\%. Such impressive results were possible due to the
filtering heuristic based on bitmaps that we proposed, which was able to reduce the generation of cluster disks in more
than 96\%, directly reducing the amount of data that needed to be processed by the algorithm. Despite the great results
achieved with our solution, we realized that there was still room for improvements, given the large availability of
multi-core processors in today's computers. With that in mind, we remodeled our proposed solution, proving how flexible
our proposed architecture was, to perform some critical and time consuming tasks in parallel, taking full advantage of
the multi-core paradigm. Our performance benchmarks showed that we could outperform our own single threaded solution by
96\% in some cases. So, to put that in numbers, we were able to reduce a running time of 15 thousand seconds
(state-of-the-art techniques) to only 13 seconds (\ac{bitdf} \ac{mt}). It is also important to mention that our
experiments were performed using various datasets, both synthetic generated and collected from real world experiments
and all of them having a considerable number of entries.

\section{Future Work}
Even though we ere able to present great contributions in this field, there are still some gaps that can be filled and
points that need more work and can lead to important results as well. Here is a list with some of them:

\begin{enumerate}
    \item The disk superset and duplicate check is very costly and would need further investigation in order to make our
        proposed architecture even more efficient. Additionally, because it modifies the same disk set it is really hard
        to parallelize it without causing performance degradation.
    \item When investigating a grid cell, its extended grid might have cells that were previously processed by other
        extended grids. Thus, we could avoid paring the same points and generating the same disks again if we could keep
        some sort of disk cache and reuse those disks.
    \item There is no study to see how different flocks relate to each other, like some \ac{mo} that starts in one flock
        and later moves to a different one.
    \item Make each module of the proposed architecture run in its own thread/process and then making it independent of
        the time spent in other modules.
\end{enumerate}

In this dissertation we showed how \ac{bitdf} and \ac{bitdf} \ac{mt} behaves when applied to datasets collected from
vehicle mobility in road networks, or datasets that simulate that same scenario. There are still other mobility
scenarios that can also take advantage of flock pattern detection:

\begin{enumerate}
    \item See how \ac{bitdf} performs in indoor flock detection.
    \item Use \ac{bitdf} in pedestrian mobility datasets.
    \item Evaluate \ac{bitdf} in animal mobility datasets, such as those from Movebank Database \citep{movebank}.
\end{enumerate}

Finally, some other data handling techniques can be used to help in the clustering process of points, such as Voronoi
Diagrams used in conjuntion with Delaunay Triangulation and data structures like K-D-Tress.

\subsection{Deployment Challenges in Real World Scenarios}
In order to be ready to be deployed and used as data mining solution, some work is still needed. In our experiments, we
noticed that, due to the nature of real world datasets, one big challenge is still the time synchronization of the data
stream, i.e. the discovery of the correct time slot size. As we said in \chapref{chp:techbackground}, we cannot assume
that the data entries in a spatio-temporal dataset are sampled in a fixed time rate, therefore there is still research
effort needed in order to synchronize and group the entries in buckets of a pre-defined size, as we did with our
$\sigma$ parameter. Therefore, for online analysis, there is a need for an automatic process to detect that $\sigma$
parameter according to the dataset being analyzed and also recalibrate that if the data behavior changes. Additionally,
such automatic process should be smart enough to deal with abnormalities, such as outliers.

Another scenario that needs to be evaluated is when we have a high speed online stream of data. With high speed incoming
data, we can face some data drop, because of a Data Connector component being busy processing the data or forwarding the
data to other components and that taking too long. Some evaluation would need to be done in that scenario and mitigation
solutions would need to be proposed. Some possible solutions to avoid data drop could be: implement some buffering
mechanism of incoming data, data sampling, data hashing, etc.

\ac{gps} data is known to be very noisy and thus generate a lot of outliers. That said, it is very important to know how
to deal with such problematic data without excluding good data entries. More importantly, when filtering bad data
entries, such pre-processing should be done in a way that does not compromise the accuracy nor the efficiency of the
solution proposed in this dissertation.
