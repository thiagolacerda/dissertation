\chapter{Technical Concepts}
\label{chp:techconcepts}
For the datasets that we use in our experiments, we assume that they are composed by a set of tuples $\{O_{id}, \phi,
\lambda, t\}$.  Where: $O_{id}$ uniquely identifies a Moving Object (MO); $t$ corresponds to the time stamp that the GPS
position was extracted; $\phi$ and $\lambda$ corresponds to the latitude and longitude of the MO respectively, at a
given time stamp $t$. Each MO represented by a $O_{id}$ has a trajectory $T_{id}$ associated with it.

\noindent
\textit{\textbf{Definition 1 (Trajectory)}: Given an $O_{id}$, an trajectory $T_{id}$ consists of a sequence of 3-D
points in the form of $<\hspace{-0.2cm}(\phi_0, \lambda_0, t_0), (\phi_1, \lambda_1, t_1), ..., (\phi_n, \lambda_n, t_n)
\hspace{-0.2cm}>$, with $n \in \mathbb{N}$ and $t_0 < t_1 < ... < t_n$.}

It is worth noting that real world datasets do not guarantee that all points of all trajectories are sampled in the same
rate. Therefore, we divide the time extent that the points appear in buckets of size $\sigma$ where $\sigma$ should be
chosen accordingly to the dataset being analyzed, based on the sampling rate of points. Hence, when we refer to any time
instance $t_i$ we are actually talking about a time bucket (or time slot) of size $\sigma$.

We use the same definition of flock from Vieira et al. \cite{bib:vieira}:

\noindent
\textit{\textbf{Definition 2 (Flock)}: Given a set of trajectories $\mathcal{T}$, a minimum number of trajectories $\mu
> 1$ ($\mu \in \mathbb{N}$), a maximum distance $\epsilon > 0$ defined over the distance function $d$, and a minimum
time duration $\delta > 1$ ($\delta \in \mathbb{N}$). A flock pattern $Flock (\mu, \epsilon, \delta)$ reports all
maximal size collections $\mathcal{F}$ of trajectories where: for each $f_k$ in $\mathcal{F}$, the number of
trajectories in $f_k$ is greater or equal than $\mu$ ($|f_k| \ge \mu$) and there exist $\delta$ consecutive timestamps
such that for every $t_i \in [f_k^{t_1}...f_k^{t_1 + \delta}]$, there is a disk with center $c_k^{t_i}$ and radius
$\epsilon/2$ covering all points in $f_k^{t_i}$. That is: $\forall f_k \in \mathcal{F}, \forall t_i \in
[f_k^{t_1}...f_k^{t_1 + \delta}], \forall T_j \in f_k: |f_k | \ge \mu, d(p_j^{t_i},c_k^{t_i}) \le \epsilon/2$.}
